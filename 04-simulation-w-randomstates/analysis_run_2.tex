% Options for packages loaded elsewhere
\PassOptionsToPackage{unicode}{hyperref}
\PassOptionsToPackage{hyphens}{url}
%
\documentclass[
]{article}
\usepackage{amsmath,amssymb}
\usepackage{iftex}
\ifPDFTeX
  \usepackage[T1]{fontenc}
  \usepackage[utf8]{inputenc}
  \usepackage{textcomp} % provide euro and other symbols
\else % if luatex or xetex
  \usepackage{unicode-math} % this also loads fontspec
  \defaultfontfeatures{Scale=MatchLowercase}
  \defaultfontfeatures[\rmfamily]{Ligatures=TeX,Scale=1}
\fi
\usepackage{lmodern}
\ifPDFTeX\else
  % xetex/luatex font selection
\fi
% Use upquote if available, for straight quotes in verbatim environments
\IfFileExists{upquote.sty}{\usepackage{upquote}}{}
\IfFileExists{microtype.sty}{% use microtype if available
  \usepackage[]{microtype}
  \UseMicrotypeSet[protrusion]{basicmath} % disable protrusion for tt fonts
}{}
\makeatletter
\@ifundefined{KOMAClassName}{% if non-KOMA class
  \IfFileExists{parskip.sty}{%
    \usepackage{parskip}
  }{% else
    \setlength{\parindent}{0pt}
    \setlength{\parskip}{6pt plus 2pt minus 1pt}}
}{% if KOMA class
  \KOMAoptions{parskip=half}}
\makeatother
\usepackage{xcolor}
\usepackage[margin=1in]{geometry}
\usepackage{color}
\usepackage{fancyvrb}
\newcommand{\VerbBar}{|}
\newcommand{\VERB}{\Verb[commandchars=\\\{\}]}
\DefineVerbatimEnvironment{Highlighting}{Verbatim}{commandchars=\\\{\}}
% Add ',fontsize=\small' for more characters per line
\usepackage{framed}
\definecolor{shadecolor}{RGB}{248,248,248}
\newenvironment{Shaded}{\begin{snugshade}}{\end{snugshade}}
\newcommand{\AlertTok}[1]{\textcolor[rgb]{0.94,0.16,0.16}{#1}}
\newcommand{\AnnotationTok}[1]{\textcolor[rgb]{0.56,0.35,0.01}{\textbf{\textit{#1}}}}
\newcommand{\AttributeTok}[1]{\textcolor[rgb]{0.13,0.29,0.53}{#1}}
\newcommand{\BaseNTok}[1]{\textcolor[rgb]{0.00,0.00,0.81}{#1}}
\newcommand{\BuiltInTok}[1]{#1}
\newcommand{\CharTok}[1]{\textcolor[rgb]{0.31,0.60,0.02}{#1}}
\newcommand{\CommentTok}[1]{\textcolor[rgb]{0.56,0.35,0.01}{\textit{#1}}}
\newcommand{\CommentVarTok}[1]{\textcolor[rgb]{0.56,0.35,0.01}{\textbf{\textit{#1}}}}
\newcommand{\ConstantTok}[1]{\textcolor[rgb]{0.56,0.35,0.01}{#1}}
\newcommand{\ControlFlowTok}[1]{\textcolor[rgb]{0.13,0.29,0.53}{\textbf{#1}}}
\newcommand{\DataTypeTok}[1]{\textcolor[rgb]{0.13,0.29,0.53}{#1}}
\newcommand{\DecValTok}[1]{\textcolor[rgb]{0.00,0.00,0.81}{#1}}
\newcommand{\DocumentationTok}[1]{\textcolor[rgb]{0.56,0.35,0.01}{\textbf{\textit{#1}}}}
\newcommand{\ErrorTok}[1]{\textcolor[rgb]{0.64,0.00,0.00}{\textbf{#1}}}
\newcommand{\ExtensionTok}[1]{#1}
\newcommand{\FloatTok}[1]{\textcolor[rgb]{0.00,0.00,0.81}{#1}}
\newcommand{\FunctionTok}[1]{\textcolor[rgb]{0.13,0.29,0.53}{\textbf{#1}}}
\newcommand{\ImportTok}[1]{#1}
\newcommand{\InformationTok}[1]{\textcolor[rgb]{0.56,0.35,0.01}{\textbf{\textit{#1}}}}
\newcommand{\KeywordTok}[1]{\textcolor[rgb]{0.13,0.29,0.53}{\textbf{#1}}}
\newcommand{\NormalTok}[1]{#1}
\newcommand{\OperatorTok}[1]{\textcolor[rgb]{0.81,0.36,0.00}{\textbf{#1}}}
\newcommand{\OtherTok}[1]{\textcolor[rgb]{0.56,0.35,0.01}{#1}}
\newcommand{\PreprocessorTok}[1]{\textcolor[rgb]{0.56,0.35,0.01}{\textit{#1}}}
\newcommand{\RegionMarkerTok}[1]{#1}
\newcommand{\SpecialCharTok}[1]{\textcolor[rgb]{0.81,0.36,0.00}{\textbf{#1}}}
\newcommand{\SpecialStringTok}[1]{\textcolor[rgb]{0.31,0.60,0.02}{#1}}
\newcommand{\StringTok}[1]{\textcolor[rgb]{0.31,0.60,0.02}{#1}}
\newcommand{\VariableTok}[1]{\textcolor[rgb]{0.00,0.00,0.00}{#1}}
\newcommand{\VerbatimStringTok}[1]{\textcolor[rgb]{0.31,0.60,0.02}{#1}}
\newcommand{\WarningTok}[1]{\textcolor[rgb]{0.56,0.35,0.01}{\textbf{\textit{#1}}}}
\usepackage{graphicx}
\makeatletter
\def\maxwidth{\ifdim\Gin@nat@width>\linewidth\linewidth\else\Gin@nat@width\fi}
\def\maxheight{\ifdim\Gin@nat@height>\textheight\textheight\else\Gin@nat@height\fi}
\makeatother
% Scale images if necessary, so that they will not overflow the page
% margins by default, and it is still possible to overwrite the defaults
% using explicit options in \includegraphics[width, height, ...]{}
\setkeys{Gin}{width=\maxwidth,height=\maxheight,keepaspectratio}
% Set default figure placement to htbp
\makeatletter
\def\fps@figure{htbp}
\makeatother
\setlength{\emergencystretch}{3em} % prevent overfull lines
\providecommand{\tightlist}{%
  \setlength{\itemsep}{0pt}\setlength{\parskip}{0pt}}
\setcounter{secnumdepth}{-\maxdimen} % remove section numbering
\ifLuaTeX
  \usepackage{selnolig}  % disable illegal ligatures
\fi
\IfFileExists{bookmark.sty}{\usepackage{bookmark}}{\usepackage{hyperref}}
\IfFileExists{xurl.sty}{\usepackage{xurl}}{} % add URL line breaks if available
\urlstyle{same}
\hypersetup{
  pdftitle={Analysis},
  pdfauthor={Your Name},
  hidelinks,
  pdfcreator={LaTeX via pandoc}}

\title{Analysis}
\author{Your Name}
\date{}

\begin{document}
\maketitle

{
\setcounter{tocdepth}{2}
\tableofcontents
}
\hypertarget{simulation-with-random-states}{%
\section{Simulation with Random
States}\label{simulation-with-random-states}}

\textbf{Single object \(O\):}

A single object is a tuple of \(<s, c, f>\), where:

\begin{itemize}
\item
  The size \(s\) is sampled from a size distribution. There are three
  types of distributions: Normal, Left-Skewed, and Right-Skewed. All
  distributions are truncated within the range \([1, 30]\).

  \begin{itemize}
  \tightlist
  \item
    Normal: \(s \sim N(\mu = 15, \sigma = 7.5)\)
  \item
    Left-Skewed: \(s \sim N(\mu = 22.5, \sigma = 7.5)\)
  \item
    Right-Skewed: \(s \sim N(\mu = 7.5, \sigma = 7.5)\)
  \end{itemize}
\item
  Both color \(c\) and form \(f\) are sampled from a Bernoulli
  distribution:

  \(c\) or \(f\) \(\sim Bern(p = 0.5)\)
\end{itemize}

\textbf{Single context \(C\):}

A single context consists of \(nobj\) objects, which serve as the input
for the model to assess. It is analogous to a single trial in a
behavioral experiment.

\textbf{A referent \(r\)} is defined as follows:

\[
<s_r, c_r, f_r>: s_r = \arg\max s \in C; c_r = f_r = 1.
\]

\textbf{Modification on the core model:}

\begin{enumerate}
\def\labelenumi{\arabic{enumi}.}
\tightlist
\item
  The method for computing the threshold is now sample-based.
\item
  We added a pragmatic listener model.
\end{enumerate}

\textbf{A communicative success} is defined as follows:

Variant 1: \[
1 \iff L_1(r|u_1, C) > L_1(r|u_2, C), r \in C
\] where \(u_1\) is ``big blue'' and \(u_2\) is ``blue big''. \(L_1\),
the pragmatic listener, is defined as: \[
L_1(r|u,C) \propto S_1(u|r,C) \cdot P(r)
\]

This is equivalent to variant 2: \[
1 \iff S_1(u_1|r, C) > S_1(u_2|r, C), r \in C
\]

Variant 3: \[
1 \iff \operatorname*{argmax}_r L_1(r|u_1,C)
\]

\textbf{Single simulation run:}

With one unique parameter setting, the sample size is 10,000 (1e4). This
corresponds to the number of all single contexts (trials). The parameter
values vary across different simulation runs and are described as
follows:

\begin{itemize}
\tightlist
\item
  nobj - {[}2,6,\ldots,18{]}, step = 4. Number of objects in a single
  context, 5 values in total.
\item
  speaker - incremental speaker or global speaker. List of speaker
  models, 2 values in total.
\item
  color semval - {[}0.90,0.92,\ldots,0.99{]}, step = 0.02. Semantic
  values for color adjectives, 6 values in total.
\item
  k - {[}0.2,0.4,\ldots,0.8{]}, step = 0.2. Percentage for determining
  the threshold for size semantics, four values in total.
\item
  wf - {[}0.2,0.4,\ldots,1{]}, step = 0.2. Parameter for perceptual
  blur, five values in total.
\item
  size distribution - normal, left-skewed, and right-skewed. Size
  distribution for sampling size for a single object, 3 values in total.
\end{itemize}

This results in a total of 1800 iterations, and 18 million samples.

\textbf{Initial results:}

\begin{enumerate}
\def\labelenumi{\arabic{enumi}.}
\tightlist
\item
  Incremental speaker generally leads to less communicative success with
  size-first ordering.
\item
  The intention of manipulating the size distribution is to manipulate
  the scale of communicative efficiency of size adjectives, from low to
  high: right-skewed, normal, left-skewed. This manipulation seems not
  very effective. The intended communicative success is the deviation of
  the real size value from the threshold. The real value is max. The
  threshold is a function of max, min, and k. Given a fixed k, to
  achieve a low threshold, we must maximize max and the difference
  between max and min. The current manipulation does not effectively
  achieve this. Currently, left-skewed is the lowest, and normal and
  right-skewed are almost equal.
\item
  The higher the wf, k, and color semval are, the lower the
  communicative success.
\end{enumerate}

\hypertarget{introduction}{%
\subsection{Introduction}\label{introduction}}

This is an analysis of the data.

Load the data.

\begin{Shaded}
\begin{Highlighting}[]
\FunctionTok{rm}\NormalTok{(}\AttributeTok{list=}\FunctionTok{ls}\NormalTok{())}
\NormalTok{filename }\OtherTok{\textless{}{-}} \StringTok{"simulation\_full\_run\_2.csv"}
\NormalTok{data }\OtherTok{\textless{}{-}} \FunctionTok{read.csv}\NormalTok{(filename)}
\end{Highlighting}
\end{Shaded}

\begin{Shaded}
\begin{Highlighting}[]
\NormalTok{data}\SpecialCharTok{$}\NormalTok{nobj }\OtherTok{\textless{}{-}}\NormalTok{ data}\SpecialCharTok{$}\NormalTok{nobj }\SpecialCharTok{/} \DecValTok{10000}
\end{Highlighting}
\end{Shaded}

Set up the theme for plot.

\begin{Shaded}
\begin{Highlighting}[]
\FunctionTok{library}\NormalTok{(tidyverse)}
\end{Highlighting}
\end{Shaded}

\begin{verbatim}
## -- Attaching core tidyverse packages ------------------------ tidyverse 2.0.0 --
## v dplyr     1.1.1     v readr     2.1.4
## v forcats   1.0.0     v stringr   1.5.0
## v ggplot2   3.5.1     v tibble    3.2.1
## v lubridate 1.9.2     v tidyr     1.3.0
## v purrr     1.0.1     
## -- Conflicts ------------------------------------------ tidyverse_conflicts() --
## x dplyr::filter() masks stats::filter()
## x dplyr::lag()    masks stats::lag()
## i Use the conflicted package (<http://conflicted.r-lib.org/>) to force all conflicts to become errors
\end{verbatim}

\begin{Shaded}
\begin{Highlighting}[]
\FunctionTok{library}\NormalTok{(aida)   }\CommentTok{\# custom helpers: https://github.com/michael{-}franke/aida{-}package}
\FunctionTok{theme\_set}\NormalTok{(}\FunctionTok{theme\_aida}\NormalTok{())}

\DocumentationTok{\#\#\#\#\#\#\#\#\#\#\#\#\#\#\#\#\#\#\#\#\#\#\#\#\#\#\#\#\#\#\#\#\#\#\#\#\#\#\#\#\#\#\#\#\#\#\#\#\#\#}
\DocumentationTok{\#\# CSP{-}colors}
\DocumentationTok{\#\#\#\#\#\#\#\#\#\#\#\#\#\#\#\#\#\#\#\#\#\#\#\#\#\#\#\#\#\#\#\#\#\#\#\#\#\#\#\#\#\#\#\#\#\#\#\#\#\#}
\NormalTok{CSP\_colors }\OtherTok{=} \FunctionTok{c}\NormalTok{(}
  \StringTok{"\#7581B3"}\NormalTok{, }\StringTok{"\#99C2C2"}\NormalTok{, }\StringTok{"\#C65353"}\NormalTok{, }\StringTok{"\#E2BA78"}\NormalTok{, }\StringTok{"\#5C7457"}\NormalTok{, }\StringTok{"\#575463"}\NormalTok{,}
  \StringTok{"\#B0B7D4"}\NormalTok{, }\StringTok{"\#66A3A3"}\NormalTok{, }\StringTok{"\#DB9494"}\NormalTok{, }\StringTok{"\#D49735"}\NormalTok{, }\StringTok{"\#9BB096"}\NormalTok{, }\StringTok{"\#D4D3D9"}\NormalTok{,}
  \StringTok{"\#414C76"}\NormalTok{, }\StringTok{"\#993333"}
\NormalTok{  )}
\CommentTok{\# setting theme colors globally}
\NormalTok{scale\_colour\_discrete }\OtherTok{\textless{}{-}} \ControlFlowTok{function}\NormalTok{(...) \{}
  \FunctionTok{scale\_colour\_manual}\NormalTok{(..., }\AttributeTok{values =}\NormalTok{ CSP\_colors)}
\NormalTok{\}}
\NormalTok{scale\_fill\_discrete }\OtherTok{\textless{}{-}} \ControlFlowTok{function}\NormalTok{(...) \{}
  \FunctionTok{scale\_fill\_manual}\NormalTok{(..., }\AttributeTok{values =}\NormalTok{ CSP\_colors)}
\NormalTok{\}}
\NormalTok{CSP\_color\_names }\OtherTok{=} \FunctionTok{c}\NormalTok{(}\StringTok{"glaucous"}\NormalTok{, }\StringTok{"opal"}\NormalTok{, }\StringTok{"shimmer"}\NormalTok{, }\StringTok{"crayola"}\NormalTok{, }\StringTok{"fern"}\NormalTok{, }\StringTok{"independence"}\NormalTok{,}
           \StringTok{"glaucous light 2"}\NormalTok{, }\StringTok{"opal dark 2"}\NormalTok{, }\StringTok{"shimmer light 2"}\NormalTok{, }\StringTok{"crayola dark 2"}\NormalTok{, }\StringTok{"fern light 3"}\NormalTok{, }\StringTok{"independence light 4"}\NormalTok{,}
           \StringTok{"glaucous dark 3"}\NormalTok{, }\StringTok{"shimmer dark 2"}
\NormalTok{           )}
\end{Highlighting}
\end{Shaded}

Display the difference between two speakers and two ordering.

\begin{Shaded}
\begin{Highlighting}[]
\NormalTok{data }\SpecialCharTok{\%\textgreater{}\%} 
  \FunctionTok{ggplot}\NormalTok{(}\FunctionTok{aes}\NormalTok{(}\AttributeTok{x=}\NormalTok{speaker, }\AttributeTok{y=}\NormalTok{probs\_big\_blue)) }\SpecialCharTok{+} 
  \CommentTok{\#geom\_point(stat="identity", position=position\_dodge()) + }
  \FunctionTok{stat\_summary}\NormalTok{(}\AttributeTok{fun.data =} \StringTok{"mean\_cl\_boot"}\NormalTok{, }\FunctionTok{aes}\NormalTok{(}\AttributeTok{y =}\NormalTok{ probs\_big\_blue), }\AttributeTok{color =}\NormalTok{ CSP\_colors[}\DecValTok{1}\NormalTok{]) }\SpecialCharTok{+}
  \FunctionTok{stat\_summary}\NormalTok{(}\AttributeTok{fun.data =} \StringTok{"mean\_cl\_boot"}\NormalTok{, }\FunctionTok{aes}\NormalTok{(}\AttributeTok{y =}\NormalTok{ probs\_blue\_big), }\AttributeTok{color =}\NormalTok{ CSP\_colors[}\DecValTok{3}\NormalTok{]) }\SpecialCharTok{+}
  \FunctionTok{scale\_y\_continuous}\NormalTok{(}\AttributeTok{sec.axis =} \FunctionTok{sec\_axis}\NormalTok{(}\SpecialCharTok{\textasciitilde{}}\NormalTok{ .)) }\SpecialCharTok{+}
  \FunctionTok{theme\_aida}\NormalTok{()}
\end{Highlighting}
\end{Shaded}

\includegraphics{analysis_run_2_files/figure-latex/unnamed-chunk-4-1.pdf}
Display the error bar for big blue.

\begin{Shaded}
\begin{Highlighting}[]
\NormalTok{data }\SpecialCharTok{\%\textgreater{}\%} 
  \FunctionTok{ggplot}\NormalTok{(}\FunctionTok{aes}\NormalTok{(}\AttributeTok{x=}\NormalTok{speaker)) }\SpecialCharTok{+} 
  \CommentTok{\#geom\_point(stat="identity", position=position\_dodge()) + }
  \FunctionTok{stat\_summary}\NormalTok{(}\AttributeTok{fun.data =} \StringTok{"mean\_cl\_boot"}\NormalTok{, }\FunctionTok{aes}\NormalTok{(}\AttributeTok{y =}\NormalTok{ probs\_big\_blue), }\AttributeTok{color =}\NormalTok{ CSP\_colors[}\DecValTok{1}\NormalTok{]) }\SpecialCharTok{+}
  \FunctionTok{scale\_y\_continuous}\NormalTok{(}\AttributeTok{sec.axis =} \FunctionTok{sec\_axis}\NormalTok{(}\SpecialCharTok{\textasciitilde{}}\NormalTok{ .)) }\SpecialCharTok{+}
  \FunctionTok{theme\_aida}\NormalTok{()}
\end{Highlighting}
\end{Shaded}

\includegraphics{analysis_run_2_files/figure-latex/unnamed-chunk-5-1.pdf}

\begin{Shaded}
\begin{Highlighting}[]
\NormalTok{data }\SpecialCharTok{\%\textgreater{}\%} 
  \FunctionTok{ggplot}\NormalTok{(}\FunctionTok{aes}\NormalTok{(}\AttributeTok{x=}\NormalTok{nobj, }\AttributeTok{y=}\NormalTok{probs\_big\_blue)) }\SpecialCharTok{+} 
  \CommentTok{\#geom\_point(stat="identity", position=position\_dodge()) + }
  \FunctionTok{stat\_summary}\NormalTok{(}\AttributeTok{fun.data =} \StringTok{"mean\_cl\_boot"}\NormalTok{, }\FunctionTok{aes}\NormalTok{(}\AttributeTok{y =}\NormalTok{ probs\_big\_blue), }\AttributeTok{color =}\NormalTok{ CSP\_colors[}\DecValTok{1}\NormalTok{]) }\SpecialCharTok{+}
  \FunctionTok{stat\_summary}\NormalTok{(}\AttributeTok{fun.data =} \StringTok{"mean\_cl\_boot"}\NormalTok{, }\FunctionTok{aes}\NormalTok{(}\AttributeTok{y =}\NormalTok{ probs\_blue\_big), }\AttributeTok{color =}\NormalTok{ CSP\_colors[}\DecValTok{3}\NormalTok{]) }\SpecialCharTok{+}
  \FunctionTok{scale\_y\_continuous}\NormalTok{(}\AttributeTok{sec.axis =} \FunctionTok{sec\_axis}\NormalTok{(}\SpecialCharTok{\textasciitilde{}}\NormalTok{ .)) }\SpecialCharTok{+}
  \FunctionTok{facet\_wrap}\NormalTok{(}\SpecialCharTok{\textasciitilde{}}\NormalTok{speaker) }\SpecialCharTok{+}
  \FunctionTok{theme\_aida}\NormalTok{()}
\end{Highlighting}
\end{Shaded}

\includegraphics{analysis_run_2_files/figure-latex/unnamed-chunk-6-1.pdf}

\begin{Shaded}
\begin{Highlighting}[]
\NormalTok{data }\SpecialCharTok{\%\textgreater{}\%} 
  \FunctionTok{ggplot}\NormalTok{(}\FunctionTok{aes}\NormalTok{(}\AttributeTok{x=}\NormalTok{color\_semvalue, }\AttributeTok{y=}\NormalTok{probs\_big\_blue)) }\SpecialCharTok{+} 
  \CommentTok{\#geom\_point(stat="identity", position=position\_dodge()) + }
  \FunctionTok{stat\_summary}\NormalTok{(}\AttributeTok{fun.data =} \StringTok{"mean\_cl\_boot"}\NormalTok{, }\FunctionTok{aes}\NormalTok{(}\AttributeTok{y =}\NormalTok{ probs\_big\_blue), }\AttributeTok{color =}\NormalTok{ CSP\_colors[}\DecValTok{1}\NormalTok{]) }\SpecialCharTok{+}
  \FunctionTok{stat\_summary}\NormalTok{(}\AttributeTok{fun.data =} \StringTok{"mean\_cl\_boot"}\NormalTok{, }\FunctionTok{aes}\NormalTok{(}\AttributeTok{y =}\NormalTok{ probs\_blue\_big), }\AttributeTok{color =}\NormalTok{ CSP\_colors[}\DecValTok{3}\NormalTok{]) }\SpecialCharTok{+}
  \FunctionTok{scale\_y\_continuous}\NormalTok{(}\AttributeTok{sec.axis =} \FunctionTok{sec\_axis}\NormalTok{(}\SpecialCharTok{\textasciitilde{}}\NormalTok{ .)) }\SpecialCharTok{+}
  \FunctionTok{facet\_grid}\NormalTok{(nobj}\SpecialCharTok{\textasciitilde{}}\NormalTok{speaker) }\SpecialCharTok{+}
  \FunctionTok{theme}\NormalTok{(}\AttributeTok{axis.text.x =} \FunctionTok{element\_text}\NormalTok{(}\AttributeTok{angle =} \DecValTok{90}\NormalTok{, }\AttributeTok{hjust =} \DecValTok{1}\NormalTok{)) }\SpecialCharTok{+}
  \FunctionTok{theme\_aida}\NormalTok{()}
\end{Highlighting}
\end{Shaded}

\includegraphics{analysis_run_2_files/figure-latex/unnamed-chunk-7-1.pdf}

\begin{Shaded}
\begin{Highlighting}[]
\CommentTok{\# descriptive statistics}
\NormalTok{data }\SpecialCharTok{\%\textgreater{}\%} 
  \FunctionTok{group\_by}\NormalTok{(speaker, nobj) }\SpecialCharTok{\%\textgreater{}\%} 
  \FunctionTok{summarise}\NormalTok{(}
    \AttributeTok{mean =} \FunctionTok{mean}\NormalTok{(probs\_big\_blue),}
    \AttributeTok{sd =} \FunctionTok{sd}\NormalTok{(probs\_big\_blue),}
    \AttributeTok{n =} \FunctionTok{n}\NormalTok{()}
\NormalTok{  )}
\end{Highlighting}
\end{Shaded}

\begin{verbatim}
## `summarise()` has grouped output by 'speaker'. You can override using the
## `.groups` argument.
\end{verbatim}

\begin{verbatim}
## # A tibble: 10 x 5
## # Groups:   speaker [2]
##    speaker              nobj   mean     sd     n
##    <chr>               <dbl>  <dbl>  <dbl> <int>
##  1 global_speaker          2 0.600  0.104  50000
##  2 global_speaker          6 0.226  0.0499 50000
##  3 global_speaker         10 0.136  0.0244 50000
##  4 global_speaker         14 0.100  0.0179 50000
##  5 global_speaker         18 0.0752 0.0128 50000
##  6 incremental_speaker     2 0.567  0.0938 50000
##  7 incremental_speaker     6 0.201  0.0389 50000
##  8 incremental_speaker    10 0.124  0.0227 50000
##  9 incremental_speaker    14 0.0893 0.0167 50000
## 10 incremental_speaker    18 0.0670 0.0123 50000
\end{verbatim}

\hypertarget{legacy-first-test-run-with-coded-comminicative-sucess}{%
\subsection{Legacy: First test run with coded comminicative
sucess}\label{legacy-first-test-run-with-coded-comminicative-sucess}}

Load the data.

\begin{Shaded}
\begin{Highlighting}[]
\NormalTok{filename }\OtherTok{\textless{}{-}} \StringTok{"simulation\_test\_run\_fewer\_parameters.csv"}
\NormalTok{data }\OtherTok{\textless{}{-}} \FunctionTok{read.csv}\NormalTok{(filename)}
\end{Highlighting}
\end{Shaded}

Set up the theme for plot.

\begin{Shaded}
\begin{Highlighting}[]
\FunctionTok{library}\NormalTok{(tidyverse)}
\FunctionTok{library}\NormalTok{(aida)   }\CommentTok{\# custom helpers: https://github.com/michael{-}franke/aida{-}package}
\FunctionTok{theme\_set}\NormalTok{(}\FunctionTok{theme\_aida}\NormalTok{())}

\DocumentationTok{\#\#\#\#\#\#\#\#\#\#\#\#\#\#\#\#\#\#\#\#\#\#\#\#\#\#\#\#\#\#\#\#\#\#\#\#\#\#\#\#\#\#\#\#\#\#\#\#\#\#}
\DocumentationTok{\#\# CSP{-}colors}
\DocumentationTok{\#\#\#\#\#\#\#\#\#\#\#\#\#\#\#\#\#\#\#\#\#\#\#\#\#\#\#\#\#\#\#\#\#\#\#\#\#\#\#\#\#\#\#\#\#\#\#\#\#\#}
\NormalTok{CSP\_colors }\OtherTok{=} \FunctionTok{c}\NormalTok{(}
  \StringTok{"\#7581B3"}\NormalTok{, }\StringTok{"\#99C2C2"}\NormalTok{, }\StringTok{"\#C65353"}\NormalTok{, }\StringTok{"\#E2BA78"}\NormalTok{, }\StringTok{"\#5C7457"}\NormalTok{, }\StringTok{"\#575463"}\NormalTok{,}
  \StringTok{"\#B0B7D4"}\NormalTok{, }\StringTok{"\#66A3A3"}\NormalTok{, }\StringTok{"\#DB9494"}\NormalTok{, }\StringTok{"\#D49735"}\NormalTok{, }\StringTok{"\#9BB096"}\NormalTok{, }\StringTok{"\#D4D3D9"}\NormalTok{,}
  \StringTok{"\#414C76"}\NormalTok{, }\StringTok{"\#993333"}
\NormalTok{  )}
\CommentTok{\# setting theme colors globally}
\NormalTok{scale\_colour\_discrete }\OtherTok{\textless{}{-}} \ControlFlowTok{function}\NormalTok{(...) \{}
  \FunctionTok{scale\_colour\_manual}\NormalTok{(..., }\AttributeTok{values =}\NormalTok{ CSP\_colors)}
\NormalTok{\}}
\NormalTok{scale\_fill\_discrete }\OtherTok{\textless{}{-}} \ControlFlowTok{function}\NormalTok{(...) \{}
  \FunctionTok{scale\_fill\_manual}\NormalTok{(..., }\AttributeTok{values =}\NormalTok{ CSP\_colors)}
\NormalTok{\}}
\NormalTok{CSP\_color\_names }\OtherTok{=} \FunctionTok{c}\NormalTok{(}\StringTok{"glaucous"}\NormalTok{, }\StringTok{"opal"}\NormalTok{, }\StringTok{"shimmer"}\NormalTok{, }\StringTok{"crayola"}\NormalTok{, }\StringTok{"fern"}\NormalTok{, }\StringTok{"independence"}\NormalTok{,}
           \StringTok{"glaucous light 2"}\NormalTok{, }\StringTok{"opal dark 2"}\NormalTok{, }\StringTok{"shimmer light 2"}\NormalTok{, }\StringTok{"crayola dark 2"}\NormalTok{, }\StringTok{"fern light 3"}\NormalTok{, }\StringTok{"independence light 4"}\NormalTok{,}
           \StringTok{"glaucous dark 3"}\NormalTok{, }\StringTok{"shimmer dark 2"}
\NormalTok{           )}
\end{Highlighting}
\end{Shaded}

\hypertarget{plot-the-propotion-of-sucess-again-color_semvalue-speaker}{%
\subsection{Plot the propotion of sucess again color\_semvalue \&
speaker}\label{plot-the-propotion-of-sucess-again-color_semvalue-speaker}}

\begin{Shaded}
\begin{Highlighting}[]
\NormalTok{data }\SpecialCharTok{\%\textgreater{}\%} 
  \FunctionTok{ggplot}\NormalTok{(}\FunctionTok{aes}\NormalTok{(}\AttributeTok{x=}\NormalTok{speaker, }\AttributeTok{y=}\NormalTok{proportion\_success)) }\SpecialCharTok{+} 
  \CommentTok{\#geom\_point(stat="identity", position=position\_dodge()) + }
  \FunctionTok{stat\_summary}\NormalTok{(}\AttributeTok{fun.data =} \StringTok{"mean\_cl\_boot"}\NormalTok{) }\SpecialCharTok{+}
  \FunctionTok{theme\_aida}\NormalTok{()}
\end{Highlighting}
\end{Shaded}

\includegraphics{analysis_run_2_files/figure-latex/unnamed-chunk-11-1.pdf}

\begin{Shaded}
\begin{Highlighting}[]
\NormalTok{data }\SpecialCharTok{\%\textgreater{}\%}
  \FunctionTok{ggplot}\NormalTok{(}\FunctionTok{aes}\NormalTok{(}\AttributeTok{x=}\NormalTok{size\_distribution, }\AttributeTok{y=}\NormalTok{proportion\_success)) }\SpecialCharTok{+} 
  \CommentTok{\#geom\_point(stat="identity", position=position\_dodge()) + }
  \FunctionTok{stat\_summary}\NormalTok{(}\AttributeTok{fun.data =} \StringTok{"mean\_cl\_boot"}\NormalTok{) }\SpecialCharTok{+}
  \FunctionTok{facet\_wrap}\NormalTok{(}\SpecialCharTok{\textasciitilde{}}\NormalTok{speaker) }\SpecialCharTok{+} 
  \FunctionTok{theme\_aida}\NormalTok{()}
\end{Highlighting}
\end{Shaded}

\includegraphics{analysis_run_2_files/figure-latex/unnamed-chunk-12-1.pdf}

To gain a basic understanding of the data and the impact of parameter
values.

\begin{Shaded}
\begin{Highlighting}[]
\NormalTok{data }\SpecialCharTok{\%\textgreater{}\%} \FunctionTok{filter}\NormalTok{(size\_distribution }\SpecialCharTok{==} \StringTok{"normal"}\NormalTok{) }\SpecialCharTok{\%\textgreater{}\%}
  \FunctionTok{ggplot}\NormalTok{(}\FunctionTok{aes}\NormalTok{(}\AttributeTok{x=}\NormalTok{color\_semvalue, }\AttributeTok{y=}\NormalTok{proportion\_success)) }\SpecialCharTok{+} 
  \CommentTok{\#geom\_point(stat="identity", position=position\_dodge()) + }
  \FunctionTok{stat\_summary}\NormalTok{(}\AttributeTok{fun.data =} \StringTok{"mean\_cl\_boot"}\NormalTok{) }\SpecialCharTok{+}
  \FunctionTok{facet\_wrap}\NormalTok{(}\SpecialCharTok{\textasciitilde{}}\NormalTok{speaker) }\SpecialCharTok{+} 
  \FunctionTok{theme\_aida}\NormalTok{()}
\end{Highlighting}
\end{Shaded}

\includegraphics{analysis_run_2_files/figure-latex/unnamed-chunk-13-1.pdf}
\#\# Plot the propotion of sucess again nobj \& speaker

\begin{Shaded}
\begin{Highlighting}[]
\NormalTok{data }\SpecialCharTok{\%\textgreater{}\%} \FunctionTok{filter}\NormalTok{(size\_distribution }\SpecialCharTok{==} \StringTok{"normal"}\NormalTok{) }\SpecialCharTok{\%\textgreater{}\%}
  \FunctionTok{ggplot}\NormalTok{(}\FunctionTok{aes}\NormalTok{(}\AttributeTok{x=}\NormalTok{nobj, }\AttributeTok{y=}\NormalTok{proportion\_success)) }\SpecialCharTok{+} 
  \CommentTok{\#geom\_point(stat="identity", position=position\_dodge()) + }
  \FunctionTok{stat\_summary}\NormalTok{(}\AttributeTok{fun.data =} \StringTok{"mean\_cl\_boot"}\NormalTok{) }\SpecialCharTok{+}
  \FunctionTok{facet\_wrap}\NormalTok{(}\SpecialCharTok{\textasciitilde{}}\NormalTok{speaker) }\SpecialCharTok{+} 
  \FunctionTok{theme\_aida}\NormalTok{()}
\end{Highlighting}
\end{Shaded}

\includegraphics{analysis_run_2_files/figure-latex/unnamed-chunk-14-1.pdf}

\begin{Shaded}
\begin{Highlighting}[]
\NormalTok{data }\SpecialCharTok{\%\textgreater{}\%} \FunctionTok{filter}\NormalTok{(size\_distribution }\SpecialCharTok{==} \StringTok{"normal"}\NormalTok{) }\SpecialCharTok{\%\textgreater{}\%}
  \FunctionTok{ggplot}\NormalTok{(}\FunctionTok{aes}\NormalTok{(}\AttributeTok{x=}\NormalTok{wf, }\AttributeTok{y=}\NormalTok{proportion\_success)) }\SpecialCharTok{+} 
  \CommentTok{\#geom\_point(stat="identity", position=position\_dodge()) + }
  \FunctionTok{stat\_summary}\NormalTok{(}\AttributeTok{fun.data =} \StringTok{"mean\_cl\_boot"}\NormalTok{) }\SpecialCharTok{+}
  \FunctionTok{facet\_wrap}\NormalTok{(}\SpecialCharTok{\textasciitilde{}}\NormalTok{speaker) }\SpecialCharTok{+} 
  \FunctionTok{theme\_aida}\NormalTok{()}
\end{Highlighting}
\end{Shaded}

\includegraphics{analysis_run_2_files/figure-latex/unnamed-chunk-15-1.pdf}

\begin{Shaded}
\begin{Highlighting}[]
\NormalTok{data }\SpecialCharTok{\%\textgreater{}\%} \FunctionTok{filter}\NormalTok{(size\_distribution }\SpecialCharTok{==} \StringTok{"normal"}\NormalTok{) }\SpecialCharTok{\%\textgreater{}\%}
  \FunctionTok{ggplot}\NormalTok{(}\FunctionTok{aes}\NormalTok{(}\AttributeTok{x=}\NormalTok{k, }\AttributeTok{y=}\NormalTok{proportion\_success)) }\SpecialCharTok{+} 
  \CommentTok{\#geom\_point(stat="identity", position=position\_dodge()) + }
  \FunctionTok{stat\_summary}\NormalTok{(}\AttributeTok{fun.data =} \StringTok{"mean\_cl\_boot"}\NormalTok{) }\SpecialCharTok{+}
  \FunctionTok{facet\_wrap}\NormalTok{(}\SpecialCharTok{\textasciitilde{}}\NormalTok{speaker) }\SpecialCharTok{+} 
  \FunctionTok{theme\_aida}\NormalTok{()}
\end{Highlighting}
\end{Shaded}

\includegraphics{analysis_run_2_files/figure-latex/unnamed-chunk-16-1.pdf}

\begin{Shaded}
\begin{Highlighting}[]
\FunctionTok{library}\NormalTok{(ggplot2)}
\CommentTok{\#ggplot(data, aes(x=color\_semvalue, y=proportion\_success)) + geom\_smooth()}
\NormalTok{data }\SpecialCharTok{\%\textgreater{}\%} \FunctionTok{filter}\NormalTok{(wf }\SpecialCharTok{==} \FloatTok{0.6} \SpecialCharTok{\&}\NormalTok{ k }\SpecialCharTok{==} \FloatTok{0.6} \SpecialCharTok{\&}\NormalTok{ size\_distribution }\SpecialCharTok{==} \StringTok{"normal"} \SpecialCharTok{\&}\NormalTok{ nobj }\SpecialCharTok{==} \DecValTok{10}\NormalTok{) }\SpecialCharTok{\%\textgreater{}\%}
  \FunctionTok{ggplot}\NormalTok{(}\FunctionTok{aes}\NormalTok{(}\AttributeTok{x=}\NormalTok{nobj, }\AttributeTok{y=}\NormalTok{proportion\_success)) }\SpecialCharTok{+} 
  \CommentTok{\#geom\_point(stat="identity", position=position\_dodge()) + }
  \FunctionTok{stat\_summary}\NormalTok{(}\AttributeTok{fun.data =} \StringTok{"mean\_cl\_boot"}\NormalTok{) }\SpecialCharTok{+} 
  \CommentTok{\#facet\_wrap(\textasciitilde{}speaker) + }
  \FunctionTok{facet\_grid}\NormalTok{(speaker}\SpecialCharTok{\textasciitilde{}}\NormalTok{color\_semvalue) }\SpecialCharTok{+}
  \FunctionTok{theme\_aida}\NormalTok{()}
\end{Highlighting}
\end{Shaded}

\begin{verbatim}
## Warning: Removed 1 row containing missing values or values outside the scale range
## (`geom_segment()`).
## Removed 1 row containing missing values or values outside the scale range
## (`geom_segment()`).
## Removed 1 row containing missing values or values outside the scale range
## (`geom_segment()`).
## Removed 1 row containing missing values or values outside the scale range
## (`geom_segment()`).
## Removed 1 row containing missing values or values outside the scale range
## (`geom_segment()`).
## Removed 1 row containing missing values or values outside the scale range
## (`geom_segment()`).
## Removed 1 row containing missing values or values outside the scale range
## (`geom_segment()`).
## Removed 1 row containing missing values or values outside the scale range
## (`geom_segment()`).
## Removed 1 row containing missing values or values outside the scale range
## (`geom_segment()`).
## Removed 1 row containing missing values or values outside the scale range
## (`geom_segment()`).
\end{verbatim}

\includegraphics{analysis_run_2_files/figure-latex/unnamed-chunk-17-1.pdf}

\end{document}
